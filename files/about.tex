\documentclass[11pt]{article}
%Gummi|065|=)
\title{\textbf{Hestia}}
\author{Sriharsh Bhyravajjula and Isha Mangurkar}
\date{\today}
\begin{document}

\maketitle

\section{About Hestia}

Hestia is a Web2py application which acts as a private image-enabled blog entry platform as well as an event manager. It has been constructed using SQL databases, HTML, CSS, JavaScript and inbuilt Web2py functions, with the documentation being carried out using Latex. It has been hosted on pythonanywhere.com


It has a homepage which introduces the functions of the application, along with an elaboration on the idea behind the project. The menubar initially consists of a link to the homepage on the top-left, with a log-in menu on the top-right, which leads to a login/sign-up page. Once signed in, access is provided for four more menus, two for entering diary-entry/images and events, and two for showing the logged in user's previous entries in both fields respectively. The entry pages have forms with validated, typeset entry fields whose valid input is stored in databases corresponding to the users. The diplay pages have an option of exporting their tabular list of contents to various formats. Each individual entry can be viewed, edited and deleted as per the given requirements.


Along with a neatly designed website, and well used functions and databases, the highlight of this application is the privacy provided to individual users as per their login. The data provided by one user cannot be accessed by any other, and hence Hestia keeps the data private with respect to each user, thus providing a neat, pleasant and productive experience to its users. 

\section{Feedback}
We hope you will enjoy using this release as much as we enjoyed creating it. If you have comments, suggestions or wish to report an issue you are experiencing - contact us at: \emph{s.bhyravajjula@research.iiit.ac.in}.



\end{document}
